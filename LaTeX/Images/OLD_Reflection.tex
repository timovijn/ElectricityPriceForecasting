\begin{subequations}
Before reaching the interface, the incident wave travels in the positive direction of the bar (where the wave speed is $c_L$). The reflected wave also travels along that part of the bar, but in negative direction. The displacement field to the left of the interface (for $x<L$) can thus be written as a sum of the displacements induced by the incident and reflected waves:

    \begin{align}
    u_L&=p(\xi_L)+g(\eta_L)
    \\[1em]
    &=p\left(t-\frac{x}{c_L}\right)+g\left(t+\frac{x}{c_L}\right)
    \end{align}
\label{eq:incident_reflected}
\end{subequations}

\begin{subequations}
While the displacement field to the right of the interface can be written as function of the transmitted wave only:
    \begin{align}
    u_R&=h(\xi_R)
    \\[1em]
    &=h\left(t-\frac{x}{c_R}\right)
    \end{align}
\label{eq:transmitted}
\end{subequations}

Where, as before, $\xi$ indicates a forward (right) moving, and $\eta$ indicates a backward (left) moving wave.

\begin{subequations}
At the bonded interface, for any $t$, we find an equal displacement left and right, and thus an equal stress:
    \begin{gather}
    u_L(L,t)=u_R(L,t)
    \\[1em]
    \sigma_L(L,t)=\sigma_R(L,t)
    \end{gather}
\end{subequations}

From Equation \ref{eq:strain_from_displacement} and \ref{eq:stress_from_strain}, we find that:
\begin{equation}
    \rho_L c_L^2 \frac{\partial{u_L}}{\partial{x}}(L,t)=\rho_R c_R^2 \frac{\partial{u_R}}{\partial{x}}(L,t)
\end{equation}

So it must hold that:
\begin{equation}
    f(t) + g(t) = h(t)
\end{equation}

Also, the value for $\delta u/\delta x$ should be equal for Equations \ref{eq:incident_reflected} and \ref{eq:transmitted} for any $t$:
\begin{subequations}
\begin{gather}
    \frac{df(t-x/c_L)}{d(t-x/c_L)}\frac{\delta(t-x/c_L)}{\delta x} + \frac{dg(t+x/c_L)}{d(t+x/c_L)}\frac{\delta(t+x/c_L)}{\delta x} = \frac{dh(t-x/c_R)}{d(t-x/c_R)}\frac{\delta(t-x/c_R)}{\delta x}
    \\[1em]
    -\frac{1}{c_L}\frac{df(t)}{dt} + \frac{1}{c_L}\frac{dg(t)}{dt} = -\frac{1}{c_R}\frac{dh(t)}{dt}
    \\[1em]
    -\frac{df(t)}{dt} + \frac{dg(t)}{dt} = -K \frac{dh(t)}{dt}
\end{gather}
\end{subequations}

Where $K$, the ratio of acoustic impedance's---in this case between the right and left of the boundary---is defined as:
\begin{equation}
    K = \frac{z_L}{z_R}
\end{equation}

where:
\begin{equation}
    z=\rho\cdot c
    \label{Eq:Acoustic_impedance}
\end{equation}

Integration of this equation leads to:
\begin{equation}
    -f(t) + g(t) = -K h(t) + C
\end{equation}

where $C$ is an integration constant.

\begin{subequations}
Finally, solving this equation for the functions $g(t)$ and $h(t)$, the reflected and transmitted waves are found as a function of the incident wave:
    \begin{gather}
        g\left(t + \frac{x}{c_L}\right) = \left(\frac{1-K}{1+K}\right) f\left(t + \frac{x}{c_L}\right)
        \\[1em]
        h\left(t - \frac{x}{c_R}\right) = \left(\frac{2}{1+K}\right) f\left(t - \frac{x}{c_R}\right)
    \end{gather}
    \label{Eq:Reflections}
\end{subequations}

Suppose that the boundary at $x=L$ is a `rigid boundary', implying that $z_R\rightarrow \infty$, and $K\rightarrow 0$ (for any value of $z_L$), we find that:
\begin{equation}
    \lim_{K \to 0} g\left(t + \frac{x}{c_L}\right) = f\left(t + \frac{x}{c_L}\right)
\end{equation}

Suppose that the boundary at $x=x_f$ is a `free boundary', implying that $z_R\rightarrow 0$, and $K\rightarrow \infty$ (for any value of $z_L$), we find that:
\begin{equation}
    \lim_{K \to \infty} g\left(t + \frac{x}{c_L}\right) = -f\left(t + \frac{x}{c_L}\right)
\end{equation}

When we introduce a new variable:

\begin{equation}
    \xi' = t - \frac{x_e - x}{c}
\end{equation}

to `simulate' a second displacement at the boundary, Equation \ref{eq:displacement_distribution} can be elaborated with a single reflection at the rigid boundary:

\begin{equation}
    u(x,t)=
\begin{cases}
    \Psi(\xi) & \text{for $\frac{x}{c} < t \leq \frac{x}{c} + \Delta t_i$}
    \\[0.5em]
    \Psi(\xi) + \Psi(\xi') & \text{for $ \frac{x_e}{c} < t \leq \frac{x_e}{c} + \Delta t_i$}
    \\[0.5em]
    \Psi(\xi') & \text{for $\frac{x_e}{c} + \Delta t_i < t \leq 2\frac{x_e}{c} + \Delta t_i$}
    \\[0.5em]
    0 & \text{for \textit{else}}
\end{cases}
\end{equation}

To better visualize the effect of the reflection, we use the boundary condition of a step function presented in Equation \ref{eq:boundary_condition_block}:
\begin{equation}
r(t)=
\begin{cases}
    \si{1e-6} & \text{for $0 < t \leq \si{1e-9}$}
    \\[0.5em]
    0 & \text{for \textit{else}}
\end{cases}
\end{equation}

amounting to a unit step, with $\Delta t_i = \SI{1}{\nano\second}$

\begin{figure}[ht]
    \centering
    \input{displacement_position_unit_step.tex}
    \caption{Visualization of the displacement distribution in the bar for every position for longitudinal displacement input $r(t)=\si{1e-6}$ at $x=\SI{0}{\metre}$\\Source: \textsc{matlab}}
\end{figure}