\begin{subequations}
Before reaching the interface, the incident wave travels in the positive direction of the bar (where the wave speed is $c_L$). The reflected wave also travels along that part of the bar, but in negative direction. The displacement field to the left of the interface (for $x<L$) can thus be written as a sum of the displacements induced by the incident and reflected waves:

    \begin{align}
    u_L&=p(\xi_L)+g(\eta_L)
    \\[1em]
    &=p\left(t-\frac{x}{c_L}\right)+g\left(t+\frac{x}{c_L}\right)
    \end{align}
\label{eq:incident_reflected}
\end{subequations}

\begin{subequations}
While the displacement field to the right of the interface can be written as function of the transmitted wave only:
    \begin{align}
    u_R&=h(\xi_R)
    \\[1em]
    &=h\left(t-\frac{x}{c_R}\right)
    \end{align}
\label{eq:transmitted}
\end{subequations}

Where, as before, $\xi$ indicates a forward (right) moving, and $\eta$ indicates a backward (left) moving wave.

\begin{subequations}
At the bonded interface, for any $t$, we find an equal displacement left and right, and thus an equal stress:
    \begin{gather}
    u_L(L,t)=u_R(L,t)
    \\[1em]
    \sigma_L(L,t)=\sigma_R(L,t)
    \end{gather}
\end{subequations}

From Equation \ref{eq:strain_from_displacement} and \ref{eq:stress_from_strain}, we find that:
\begin{equation}
    \rho_L c_L^2 \frac{\partial{u_L}}{\partial{x}}(L,t)=\rho_R c_R^2 \frac{\partial{u_R}}{\partial{x}}(L,t)
\end{equation}

So it must hold that:
\begin{equation}
    f(t) + g(t) = h(t)
\end{equation}

Also, the value for $\delta u/\delta x$ should be equal for Equations \ref{eq:incident_reflected} and \ref{eq:transmitted} for any $t$:
\begin{subequations}
\begin{gather}
    \frac{df(t-x/c_L)}{d(t-x/c_L)}\frac{\delta(t-x/c_L)}{\delta x} + \frac{dg(t+x/c_L)}{d(t+x/c_L)}\frac{\delta(t+x/c_L)}{\delta x} = \frac{dh(t-x/c_R)}{d(t-x/c_R)}\frac{\delta(t-x/c_R)}{\delta x}
    \\[1em]
    -\frac{1}{c_L}\frac{df(t)}{dt} + \frac{1}{c_L}\frac{dg(t)}{dt} = -\frac{1}{c_R}\frac{dh(t)}{dt}
    \\[1em]
    -\frac{df(t)}{dt} + \frac{dg(t)}{dt} = -K \frac{dh(t)}{dt}
\end{gather}
\end{subequations}

Where $K$, the ratio of acoustic impedance's---in this case between the right and left of the boundary---is defined as:
\begin{equation}
    K = \frac{z_L}{z_R}
\end{equation}

where:
\begin{equation}
    z=\rho\cdot c
    \label{Eq:Acoustic_impedance}
\end{equation}

Integration of this equation leads to:
\begin{equation}
    -f(t) + g(t) = -K h(t) + C
\end{equation}

where $C$ is an integration constant.

\begin{subequations}
Finally, solving this equation for the functions $g(t)$ and $h(t)$, the reflected and transmitted waves are found as a function of the incident wave:
    \begin{gather}
        g\left(t + \frac{x}{c_L}\right) = \left(\frac{1-K}{1+K}\right) f\left(t + \frac{x}{c_L}\right)
        \\[1em]
        h\left(t - \frac{x}{c_R}\right) = \left(\frac{2}{1+K}\right) f\left(t - \frac{x}{c_R}\right)
    \end{gather}
    \label{Eq:Reflections}
\end{subequations}

Suppose that the boundary at $x=L$ is a `rigid boundary', implying that $z_R\rightarrow \infty$, and $K\rightarrow 0$ (for any value of $z_L$), we find that:
\begin{equation}
    \lim_{K \to 0} g\left(t + \frac{x}{c_L}\right) = f\left(t + \frac{x}{c_L}\right)
\end{equation}

Suppose that the boundary at $x=x_f$ is a `free boundary', implying that $z_R\rightarrow 0$, and $K\rightarrow \infty$ (for any value of $z_L$), we find that:
\begin{equation}
    \lim_{K \to \infty} g\left(t + \frac{x}{c_L}\right) = -f\left(t + \frac{x}{c_L}\right)
\end{equation}

When we introduce a new variable:

\begin{equation}
    \xi' = t - \frac{x_e - x}{c}
\end{equation}

to `simulate' a second displacement at the boundary, Equation \ref{eq:displacement_distribution} can be elaborated with a single reflection at the rigid boundary:

\begin{equation}
    u(x,t)=
\begin{cases}
    \Psi(\xi) & \text{for $\frac{x}{c} < t \leq \frac{x}{c} + \Delta t_i$}
    \\[0.5em]
    \Psi(\xi) + \Psi(\xi') & \text{for $ \frac{x_e}{c} < t \leq \frac{x_e}{c} + \Delta t_i$}
    \\[0.5em]
    \Psi(\xi') & \text{for $\frac{x_e}{c} + \Delta t_i < t \leq 2\frac{x_e}{c} + \Delta t_i$}
    \\[0.5em]
    0 & \text{for \textit{else}}
\end{cases}
\end{equation}

To better visualize the effect of the reflection, we use the boundary condition of a step function presented in Equation \ref{eq:boundary_condition_block}:
\begin{equation}
r(t)=
\begin{cases}
    \si{1e-6} & \text{for $0 < t \leq \si{1e-9}$}
    \\[0.5em]
    0 & \text{for \textit{else}}
\end{cases}
\end{equation}

amounting to a unit step, with $\Delta t_i = \SI{1}{\nano\second}$

\begin{figure}[ht]
    \centering
    % This file was created by matlab2tikz.
%
%The latest updates can be retrieved from
%  http://www.mathworks.com/matlabcentral/fileexchange/22022-matlab2tikz-matlab2tikz
%where you can also make suggestions and rate matlab2tikz.
%
\definecolor{mycolor1}{rgb}{0.00000,0.44700,0.74100}%
%
\begin{tikzpicture}

\begin{axis}[%
width=100mm,
height=30mm,
at={(0.758in,0.481in)},
scale only axis,
xmin=0,
xmax=1.2,
xlabel={Position (\si{\metre})},
ymin=-3e-6,
ymax=3e-6,
ylabel={Displacement in x direction (\si{\metre})},
axis background/.style={fill=none,draw=MaterialGrey500},
xlabel style={font=\color{white!15!black}},
ylabel style={font=\color{white!15!black}},
legend style={fill=none,draw=none,font=\color{MaterialGrey500}\footnotesize},
legend pos = south east
]
\addplot [color=MaterialGrey900, line width = .2mm]
  table[row sep=crcr]{%
0	0\\
0.001	0\\
0.002	0\\
0.003	0\\
0.004	0\\
0.005	0\\
0.006	0\\
0.007	0\\
0.008	0\\
0.009	0\\
0.01	0\\
0.011	0\\
0.012	0\\
0.013	0\\
0.014	0\\
0.015	0\\
0.016	0\\
0.017	0\\
0.018	0\\
0.019	0\\
0.02	0\\
0.021	0\\
0.022	0\\
0.023	0\\
0.024	0\\
0.025	0\\
0.026	0\\
0.027	0\\
0.028	0\\
0.029	0\\
0.03	0\\
0.031	0\\
0.032	0\\
0.033	0\\
0.034	0\\
0.035	0\\
0.036	0\\
0.037	0\\
0.038	0\\
0.039	0\\
0.04	0\\
0.041	0\\
0.042	0\\
0.043	0\\
0.044	0\\
0.045	0\\
0.046	0\\
0.047	0\\
0.048	0\\
0.049	0\\
0.05	0\\
0.051	0\\
0.052	0\\
0.053	0\\
0.054	0\\
0.055	0\\
0.056	0\\
0.057	0\\
0.058	0\\
0.059	0\\
0.06	0\\
0.061	0\\
0.062	0\\
0.063	0\\
0.064	0\\
0.065	0\\
0.066	0\\
0.067	0\\
0.068	0\\
0.069	0\\
0.07	0\\
0.071	0\\
0.072	0\\
0.073	0\\
0.074	0\\
0.075	0\\
0.076	0\\
0.077	0\\
0.078	0\\
0.079	0\\
0.08	0\\
0.081	0\\
0.082	0\\
0.083	0\\
0.084	0\\
0.085	0\\
0.086	0\\
0.087	0\\
0.088	0\\
0.089	0\\
0.09	0\\
0.091	0\\
0.092	0\\
0.093	0\\
0.094	0\\
0.095	0\\
0.096	0\\
0.097	0\\
0.098	0\\
0.099	0\\
0.1	0\\
0.101	0\\
0.102	0\\
0.103	0\\
0.104	0\\
0.105	0\\
0.106	0\\
0.107	0\\
0.108	0\\
0.109	0\\
0.11	0\\
0.111	0\\
0.112	0\\
0.113	0\\
0.114	0\\
0.115	0\\
0.116	0\\
0.117	0\\
0.118	0\\
0.119	0\\
0.12	0\\
0.121	0\\
0.122	0\\
0.123	0\\
0.124	0\\
0.125	0\\
0.126	0\\
0.127	0\\
0.128	0\\
0.129	0\\
0.13	0\\
0.131	0\\
0.132	0\\
0.133	0\\
0.134	0\\
0.135	0\\
0.136	0\\
0.137	0\\
0.138	0\\
0.139	0\\
0.14	0\\
0.141	0\\
0.142	0\\
0.143	0\\
0.144	0\\
0.145	0\\
0.146	0\\
0.147	0\\
0.148	0\\
0.149	0\\
0.15	0\\
0.151	0\\
0.152	0\\
0.153	0\\
0.154	0\\
0.155	0\\
0.156	0\\
0.157	0\\
0.158	0\\
0.159	0\\
0.16	0\\
0.161	0\\
0.162	0\\
0.163	0\\
0.164	0\\
0.165	0\\
0.166	0\\
0.167	0\\
0.168	0\\
0.169	0\\
0.17	0\\
0.171	0\\
0.172	0\\
0.173	0\\
0.174	0\\
0.175	0\\
0.176	0\\
0.177	0\\
0.178	0\\
0.179	0\\
0.18	0\\
0.181	0\\
0.182	0\\
0.183	0\\
0.184	0\\
0.185	0\\
0.186	0\\
0.187	0\\
0.188	0\\
0.189	0\\
0.19	0\\
0.191	0\\
0.192	0\\
0.193	0\\
0.194	0\\
0.195	0\\
0.196	0\\
0.197	0\\
0.198	0\\
0.199	0\\
0.2	0\\
0.201	0\\
0.202	0\\
0.203	0\\
0.204	0\\
0.205	0\\
0.206	0\\
0.207	0\\
0.208	0\\
0.209	0\\
0.21	0\\
0.211	0\\
0.212	0\\
0.213	0\\
0.214	0\\
0.215	0\\
0.216	0\\
0.217	0\\
0.218	0\\
0.219	0\\
0.22	0\\
0.221	0\\
0.222	0\\
0.223	0\\
0.224	0\\
0.225	0\\
0.226	0\\
0.227	0\\
0.228	0\\
0.229	0\\
0.23	0\\
0.231	0\\
0.232	0\\
0.233	0\\
0.234	0\\
0.235	0\\
0.236	0\\
0.237	0\\
0.238	0\\
0.239	0\\
0.24	0\\
0.241	0\\
0.242	0\\
0.243	0\\
0.244	0\\
0.245	0\\
0.246	0\\
0.247	0\\
0.248	0\\
0.249	0\\
0.25	0\\
0.251	0\\
0.252	0\\
0.253	0\\
0.254	0\\
0.255	0\\
0.256	0\\
0.257	0\\
0.258	0\\
0.259	0\\
0.26	0\\
0.261	0\\
0.262	0\\
0.263	0\\
0.264	0\\
0.265	0\\
0.266	0\\
0.267	0\\
0.268	0\\
0.269	0\\
0.27	0\\
0.271	0\\
0.272	0\\
0.273	0\\
0.274	0\\
0.275	0\\
0.276	0\\
0.277	0\\
0.278	0\\
0.279	0\\
0.28	0\\
0.281	0\\
0.282	0\\
0.283	0\\
0.284	0\\
0.285	0\\
0.286	0\\
0.287	0\\
0.288	0\\
0.289	0\\
0.29	0\\
0.291	0\\
0.292	0\\
0.293	0\\
0.294	0\\
0.295	0\\
0.296	0\\
0.297	0\\
0.298	0\\
0.299	0\\
0.3	0\\
0.301	0\\
0.302	0\\
0.303	0\\
0.304	0\\
0.305	0\\
0.306	0\\
0.307	0\\
0.308	0\\
0.309	0\\
0.31	0\\
0.311	0\\
0.312	0\\
0.313	0\\
0.314	0\\
0.315	0\\
0.316	0\\
0.317	0\\
0.318	0\\
0.319	0\\
0.32	0\\
0.321	0\\
0.322	0\\
0.323	0\\
0.324	0\\
0.325	1e-06\\
0.326	1e-06\\
0.327	1e-06\\
0.328	1e-06\\
0.329	1e-06\\
0.33	1e-06\\
0.331	1e-06\\
0.332	1e-06\\
0.333	1e-06\\
0.334	1e-06\\
0.335	1e-06\\
0.336	1e-06\\
0.337	1e-06\\
0.338	1e-06\\
0.339	1e-06\\
0.34	1e-06\\
0.341	1e-06\\
0.342	1e-06\\
0.343	1e-06\\
0.344	1e-06\\
0.345	1e-06\\
0.346	1e-06\\
0.347	1e-06\\
0.348	1e-06\\
0.349	1e-06\\
0.35	1e-06\\
0.351	1e-06\\
0.352	1e-06\\
0.353	1e-06\\
0.354	1e-06\\
0.355	1e-06\\
0.356	1e-06\\
0.357	1e-06\\
0.358	1e-06\\
0.359	1e-06\\
0.36	1e-06\\
0.361	1e-06\\
0.362	1e-06\\
0.363	1e-06\\
0.364	1e-06\\
0.365	1e-06\\
0.366	1e-06\\
0.367	1e-06\\
0.368	1e-06\\
0.369	1e-06\\
0.37	1e-06\\
0.371	1e-06\\
0.372	1e-06\\
0.373	1e-06\\
0.374	1e-06\\
0.375	1e-06\\
0.376	1e-06\\
0.377	1e-06\\
0.378	1e-06\\
0.379	1e-06\\
0.38	1e-06\\
0.381	1e-06\\
0.382	1e-06\\
0.383	1e-06\\
0.384	1e-06\\
0.385	1e-06\\
0.386	1e-06\\
0.387	1e-06\\
0.388	1e-06\\
0.389	1e-06\\
0.39	1e-06\\
0.391	1e-06\\
0.392	1e-06\\
0.393	1e-06\\
0.394	1e-06\\
0.395	1e-06\\
0.396	1e-06\\
0.397	1e-06\\
0.398	1e-06\\
0.399	1e-06\\
0.4	1e-06\\
0.401	1e-06\\
0.402	1e-06\\
0.403	1e-06\\
0.404	1e-06\\
0.405	1e-06\\
0.406	1e-06\\
0.407	1e-06\\
0.408	1e-06\\
0.409	1e-06\\
0.41	1e-06\\
0.411	1e-06\\
0.412	1e-06\\
0.413	1e-06\\
0.414	1e-06\\
0.415	1e-06\\
0.416	1e-06\\
0.417	1e-06\\
0.418	1e-06\\
0.419	0\\
0.42	0\\
0.421	0\\
0.422	0\\
0.423	0\\
0.424	0\\
0.425	0\\
0.426	0\\
0.427	0\\
0.428	0\\
0.429	0\\
0.43	0\\
0.431	0\\
0.432	0\\
0.433	0\\
0.434	0\\
0.435	0\\
0.436	0\\
0.437	0\\
0.438	0\\
0.439	0\\
0.44	0\\
0.441	0\\
0.442	0\\
0.443	0\\
0.444	0\\
0.445	0\\
0.446	0\\
0.447	0\\
0.448	0\\
0.449	0\\
0.45	0\\
0.451	0\\
0.452	0\\
0.453	0\\
0.454	0\\
0.455	0\\
0.456	0\\
0.457	0\\
0.458	0\\
0.459	0\\
0.46	0\\
0.461	0\\
0.462	0\\
0.463	0\\
0.464	0\\
0.465	0\\
0.466	0\\
0.467	0\\
0.468	0\\
0.469	0\\
0.47	0\\
0.471	0\\
0.472	0\\
0.473	0\\
0.474	0\\
0.475	0\\
0.476	0\\
0.477	0\\
0.478	0\\
0.479	0\\
0.48	0\\
0.481	0\\
0.482	0\\
0.483	0\\
0.484	0\\
0.485	0\\
0.486	0\\
0.487	0\\
0.488	0\\
0.489	0\\
0.49	0\\
0.491	0\\
0.492	0\\
0.493	0\\
0.494	0\\
0.495	0\\
0.496	0\\
0.497	0\\
0.498	0\\
0.499	0\\
0.5	0\\
0.501	0\\
0.502	0\\
0.503	0\\
0.504	0\\
0.505	0\\
0.506	0\\
0.507	0\\
0.508	0\\
0.509	0\\
0.51	0\\
0.511	0\\
0.512	0\\
0.513	0\\
0.514	0\\
0.515	0\\
0.516	0\\
0.517	0\\
0.518	0\\
0.519	0\\
0.52	0\\
0.521	0\\
0.522	0\\
0.523	0\\
0.524	0\\
0.525	0\\
0.526	0\\
0.527	0\\
0.528	0\\
0.529	0\\
0.53	0\\
0.531	0\\
0.532	0\\
0.533	0\\
0.534	0\\
0.535	0\\
0.536	0\\
0.537	0\\
0.538	0\\
0.539	0\\
0.54	0\\
0.541	0\\
0.542	0\\
0.543	0\\
0.544	0\\
0.545	0\\
0.546	0\\
0.547	0\\
0.548	0\\
0.549	0\\
0.55	0\\
0.551	0\\
0.552	0\\
0.553	0\\
0.554	0\\
0.555	0\\
0.556	0\\
0.557	0\\
0.558	0\\
0.559	0\\
0.56	0\\
0.561	0\\
0.562	0\\
0.563	0\\
0.564	0\\
0.565	0\\
0.566	0\\
0.567	0\\
0.568	0\\
0.569	0\\
0.57	0\\
0.571	0\\
0.572	0\\
0.573	0\\
0.574	0\\
0.575	0\\
0.576	0\\
0.577	0\\
0.578	0\\
0.579	0\\
0.58	0\\
0.581	0\\
0.582	0\\
0.583	0\\
0.584	0\\
0.585	0\\
0.586	0\\
0.587	0\\
0.588	0\\
0.589	0\\
0.59	0\\
0.591	0\\
0.592	0\\
0.593	0\\
0.594	0\\
0.595	0\\
0.596	0\\
0.597	0\\
0.598	0\\
0.599	0\\
0.6	0\\
0.601	0\\
0.602	0\\
0.603	0\\
0.604	0\\
0.605	0\\
0.606	0\\
0.607	0\\
0.608	0\\
0.609	0\\
0.61	0\\
0.611	0\\
0.612	0\\
0.613	0\\
0.614	0\\
0.615	0\\
0.616	0\\
0.617	0\\
0.618	0\\
0.619	0\\
0.62	0\\
0.621	0\\
0.622	0\\
0.623	0\\
0.624	0\\
0.625	0\\
0.626	0\\
0.627	0\\
0.628	0\\
0.629	0\\
0.63	0\\
0.631	0\\
0.632	0\\
0.633	0\\
0.634	0\\
0.635	0\\
0.636	0\\
0.637	0\\
0.638	0\\
0.639	0\\
0.64	0\\
0.641	0\\
0.642	0\\
0.643	0\\
0.644	0\\
0.645	0\\
0.646	0\\
0.647	0\\
0.648	0\\
0.649	0\\
0.65	0\\
0.651	0\\
0.652	0\\
0.653	0\\
0.654	0\\
0.655	0\\
0.656	0\\
0.657	0\\
0.658	0\\
0.659	0\\
0.66	0\\
0.661	0\\
0.662	0\\
0.663	0\\
0.664	0\\
0.665	0\\
0.666	0\\
0.667	0\\
0.668	0\\
0.669	0\\
0.67	0\\
0.671	0\\
0.672	0\\
0.673	0\\
0.674	0\\
0.675	0\\
0.676	0\\
0.677	0\\
0.678	0\\
0.679	0\\
0.68	0\\
0.681	0\\
0.682	0\\
0.683	0\\
0.684	0\\
0.685	0\\
0.686	0\\
0.687	0\\
0.688	0\\
0.689	0\\
0.69	0\\
0.691	0\\
0.692	0\\
0.693	0\\
0.694	0\\
0.695	0\\
0.696	0\\
0.697	0\\
0.698	0\\
0.699	0\\
0.7	0\\
0.701	0\\
0.702	0\\
0.703	0\\
0.704	0\\
0.705	0\\
0.706	0\\
0.707	0\\
0.708	0\\
0.709	0\\
0.71	0\\
0.711	0\\
0.712	0\\
0.713	0\\
0.714	0\\
0.715	0\\
0.716	0\\
0.717	0\\
0.718	0\\
0.719	0\\
0.72	0\\
0.721	0\\
0.722	0\\
0.723	0\\
0.724	0\\
0.725	0\\
0.726	0\\
0.727	0\\
0.728	0\\
0.729	0\\
0.73	0\\
0.731	0\\
0.732	0\\
0.733	0\\
0.734	0\\
0.735	0\\
0.736	0\\
0.737	0\\
0.738	0\\
0.739	0\\
0.74	0\\
0.741	0\\
0.742	0\\
0.743	0\\
0.744	0\\
0.745	0\\
0.746	0\\
0.747	0\\
0.748	0\\
0.749	0\\
0.75	0\\
0.751	0\\
0.752	0\\
0.753	0\\
0.754	0\\
0.755	0\\
0.756	0\\
0.757	0\\
0.758	0\\
0.759	0\\
0.76	0\\
0.761	0\\
0.762	0\\
0.763	0\\
0.764	0\\
0.765	0\\
0.766	0\\
0.767	0\\
0.768	0\\
0.769	0\\
0.77	0\\
0.771	0\\
0.772	0\\
0.773	0\\
0.774	0\\
0.775	0\\
0.776	0\\
0.777	0\\
0.778	0\\
0.779	0\\
0.78	0\\
0.781	0\\
0.782	0\\
0.783	0\\
0.784	0\\
0.785	0\\
0.786	0\\
0.787	0\\
0.788	0\\
0.789	0\\
0.79	0\\
0.791	0\\
0.792	0\\
0.793	0\\
0.794	0\\
0.795	0\\
0.796	0\\
0.797	0\\
0.798	0\\
0.799	0\\
0.8	0\\
0.801	0\\
0.802	0\\
0.803	0\\
0.804	0\\
0.805	0\\
0.806	0\\
0.807	0\\
0.808	0\\
0.809	0\\
0.81	0\\
0.811	0\\
0.812	0\\
0.813	0\\
0.814	0\\
0.815	0\\
0.816	0\\
0.817	0\\
0.818	0\\
0.819	0\\
0.82	0\\
0.821	0\\
0.822	0\\
0.823	0\\
0.824	0\\
0.825	0\\
0.826	0\\
0.827	0\\
0.828	0\\
0.829	0\\
0.83	0\\
0.831	0\\
0.832	0\\
0.833	0\\
0.834	0\\
0.835	0\\
0.836	0\\
0.837	0\\
0.838	0\\
0.839	0\\
0.84	0\\
0.841	0\\
0.842	0\\
0.843	0\\
0.844	0\\
0.845	0\\
0.846	0\\
0.847	0\\
0.848	0\\
0.849	0\\
0.85	0\\
0.851	0\\
0.852	0\\
0.853	0\\
0.854	0\\
0.855	0\\
0.856	0\\
0.857	0\\
0.858	0\\
0.859	0\\
0.86	0\\
0.861	0\\
0.862	0\\
0.863	0\\
0.864	0\\
0.865	0\\
0.866	0\\
0.867	0\\
0.868	0\\
0.869	0\\
0.87	0\\
0.871	0\\
0.872	0\\
0.873	0\\
0.874	0\\
0.875	0\\
0.876	0\\
0.877	0\\
0.878	0\\
0.879	0\\
0.88	0\\
0.881	0\\
0.882	0\\
0.883	0\\
0.884	0\\
0.885	0\\
0.886	0\\
0.887	0\\
0.888	0\\
0.889	0\\
0.89	0\\
0.891	0\\
0.892	0\\
0.893	0\\
0.894	0\\
0.895	0\\
0.896	0\\
0.897	0\\
0.898	0\\
0.899	0\\
0.9	0\\
0.901	0\\
0.902	0\\
0.903	0\\
0.904	0\\
0.905	0\\
0.906	0\\
0.907	0\\
0.908	0\\
0.909	0\\
0.91	0\\
0.911	0\\
0.912	0\\
0.913	0\\
0.914	0\\
0.915	0\\
0.916	0\\
0.917	0\\
0.918	0\\
0.919	0\\
0.92	0\\
0.921	0\\
0.922	0\\
0.923	0\\
0.924	0\\
0.925	0\\
0.926	0\\
0.927	0\\
0.928	0\\
0.929	0\\
0.93	0\\
0.931	0\\
0.932	0\\
0.933	0\\
0.934	0\\
0.935	0\\
0.936	0\\
0.937	0\\
0.938	0\\
0.939	0\\
0.94	0\\
0.941	0\\
0.942	0\\
0.943	0\\
0.944	0\\
0.945	0\\
0.946	0\\
0.947	0\\
0.948	0\\
0.949	0\\
0.95	0\\
0.951	0\\
0.952	0\\
0.953	0\\
0.954	0\\
0.955	0\\
0.956	0\\
0.957	0\\
0.958	0\\
0.959	0\\
0.96	0\\
0.961	0\\
0.962	0\\
0.963	0\\
0.964	0\\
0.965	0\\
0.966	0\\
0.967	0\\
0.968	0\\
0.969	0\\
0.97	0\\
0.971	0\\
0.972	0\\
0.973	0\\
0.974	0\\
0.975	0\\
0.976	0\\
0.977	0\\
0.978	0\\
0.979	0\\
0.98	0\\
0.981	0\\
0.982	0\\
0.983	0\\
0.984	0\\
0.985	0\\
0.986	0\\
0.987	0\\
0.988	0\\
0.989	0\\
0.99	0\\
0.991	0\\
0.992	0\\
0.993	0\\
0.994	0\\
0.995	0\\
0.996	0\\
0.997	0\\
0.998	0\\
0.999	0\\
1	0\\
};
\addlegendentry{$t = \SI{4.42}{\nano\second}$}

\addplot [color=MaterialGrey900, line width = .2mm, dashed]
  table[row sep=crcr]{%
0	0\\
0.001	0\\
0.002	0\\
0.003	0\\
0.004	0\\
0.005	0\\
0.006	0\\
0.007	0\\
0.008	0\\
0.009	0\\
0.01	0\\
0.011	0\\
0.012	0\\
0.013	0\\
0.014	0\\
0.015	0\\
0.016	0\\
0.017	0\\
0.018	0\\
0.019	0\\
0.02	0\\
0.021	0\\
0.022	0\\
0.023	0\\
0.024	0\\
0.025	0\\
0.026	0\\
0.027	0\\
0.028	0\\
0.029	0\\
0.03	0\\
0.031	0\\
0.032	0\\
0.033	0\\
0.034	0\\
0.035	0\\
0.036	0\\
0.037	0\\
0.038	0\\
0.039	0\\
0.04	0\\
0.041	0\\
0.042	0\\
0.043	0\\
0.044	0\\
0.045	0\\
0.046	0\\
0.047	0\\
0.048	0\\
0.049	0\\
0.05	0\\
0.051	0\\
0.052	0\\
0.053	0\\
0.054	0\\
0.055	0\\
0.056	0\\
0.057	0\\
0.058	0\\
0.059	0\\
0.06	0\\
0.061	0\\
0.062	0\\
0.063	0\\
0.064	0\\
0.065	0\\
0.066	0\\
0.067	0\\
0.068	0\\
0.069	0\\
0.07	0\\
0.071	0\\
0.072	0\\
0.073	0\\
0.074	0\\
0.075	0\\
0.076	0\\
0.077	0\\
0.078	0\\
0.079	0\\
0.08	0\\
0.081	0\\
0.082	0\\
0.083	0\\
0.084	0\\
0.085	0\\
0.086	0\\
0.087	0\\
0.088	0\\
0.089	0\\
0.09	0\\
0.091	0\\
0.092	0\\
0.093	0\\
0.094	0\\
0.095	0\\
0.096	0\\
0.097	0\\
0.098	0\\
0.099	0\\
0.1	0\\
0.101	0\\
0.102	0\\
0.103	0\\
0.104	0\\
0.105	0\\
0.106	0\\
0.107	0\\
0.108	0\\
0.109	0\\
0.11	0\\
0.111	0\\
0.112	0\\
0.113	0\\
0.114	0\\
0.115	0\\
0.116	0\\
0.117	0\\
0.118	0\\
0.119	0\\
0.12	0\\
0.121	0\\
0.122	0\\
0.123	0\\
0.124	0\\
0.125	0\\
0.126	0\\
0.127	0\\
0.128	0\\
0.129	0\\
0.13	0\\
0.131	0\\
0.132	0\\
0.133	0\\
0.134	0\\
0.135	0\\
0.136	0\\
0.137	0\\
0.138	0\\
0.139	0\\
0.14	0\\
0.141	0\\
0.142	0\\
0.143	0\\
0.144	0\\
0.145	0\\
0.146	0\\
0.147	0\\
0.148	0\\
0.149	0\\
0.15	0\\
0.151	0\\
0.152	0\\
0.153	0\\
0.154	0\\
0.155	0\\
0.156	0\\
0.157	0\\
0.158	0\\
0.159	0\\
0.16	0\\
0.161	0\\
0.162	0\\
0.163	0\\
0.164	0\\
0.165	0\\
0.166	0\\
0.167	0\\
0.168	0\\
0.169	0\\
0.17	0\\
0.171	0\\
0.172	0\\
0.173	0\\
0.174	0\\
0.175	0\\
0.176	0\\
0.177	0\\
0.178	0\\
0.179	0\\
0.18	0\\
0.181	0\\
0.182	0\\
0.183	0\\
0.184	0\\
0.185	0\\
0.186	0\\
0.187	0\\
0.188	0\\
0.189	0\\
0.19	0\\
0.191	0\\
0.192	0\\
0.193	0\\
0.194	0\\
0.195	0\\
0.196	0\\
0.197	0\\
0.198	0\\
0.199	0\\
0.2	0\\
0.201	0\\
0.202	0\\
0.203	0\\
0.204	0\\
0.205	0\\
0.206	0\\
0.207	0\\
0.208	0\\
0.209	0\\
0.21	0\\
0.211	0\\
0.212	0\\
0.213	0\\
0.214	0\\
0.215	0\\
0.216	0\\
0.217	0\\
0.218	0\\
0.219	0\\
0.22	0\\
0.221	0\\
0.222	0\\
0.223	0\\
0.224	0\\
0.225	0\\
0.226	0\\
0.227	0\\
0.228	0\\
0.229	0\\
0.23	0\\
0.231	0\\
0.232	0\\
0.233	0\\
0.234	0\\
0.235	0\\
0.236	0\\
0.237	0\\
0.238	0\\
0.239	0\\
0.24	0\\
0.241	0\\
0.242	0\\
0.243	0\\
0.244	0\\
0.245	0\\
0.246	0\\
0.247	0\\
0.248	0\\
0.249	0\\
0.25	0\\
0.251	0\\
0.252	0\\
0.253	0\\
0.254	0\\
0.255	0\\
0.256	0\\
0.257	0\\
0.258	0\\
0.259	0\\
0.26	0\\
0.261	0\\
0.262	0\\
0.263	0\\
0.264	0\\
0.265	0\\
0.266	0\\
0.267	0\\
0.268	0\\
0.269	0\\
0.27	0\\
0.271	0\\
0.272	0\\
0.273	0\\
0.274	0\\
0.275	0\\
0.276	0\\
0.277	0\\
0.278	0\\
0.279	0\\
0.28	0\\
0.281	0\\
0.282	0\\
0.283	0\\
0.284	0\\
0.285	0\\
0.286	0\\
0.287	0\\
0.288	0\\
0.289	0\\
0.29	0\\
0.291	0\\
0.292	0\\
0.293	0\\
0.294	0\\
0.295	0\\
0.296	0\\
0.297	0\\
0.298	0\\
0.299	0\\
0.3	0\\
0.301	0\\
0.302	0\\
0.303	0\\
0.304	0\\
0.305	0\\
0.306	0\\
0.307	0\\
0.308	0\\
0.309	0\\
0.31	0\\
0.311	0\\
0.312	0\\
0.313	0\\
0.314	0\\
0.315	0\\
0.316	0\\
0.317	0\\
0.318	0\\
0.319	0\\
0.32	0\\
0.321	0\\
0.322	0\\
0.323	0\\
0.324	0\\
0.325	0\\
0.326	0\\
0.327	0\\
0.328	0\\
0.329	0\\
0.33	0\\
0.331	0\\
0.332	0\\
0.333	0\\
0.334	0\\
0.335	0\\
0.336	0\\
0.337	0\\
0.338	0\\
0.339	0\\
0.34	0\\
0.341	0\\
0.342	0\\
0.343	0\\
0.344	0\\
0.345	0\\
0.346	0\\
0.347	0\\
0.348	0\\
0.349	0\\
0.35	0\\
0.351	0\\
0.352	0\\
0.353	0\\
0.354	0\\
0.355	0\\
0.356	0\\
0.357	0\\
0.358	0\\
0.359	0\\
0.36	0\\
0.361	0\\
0.362	0\\
0.363	0\\
0.364	0\\
0.365	0\\
0.366	0\\
0.367	0\\
0.368	0\\
0.369	0\\
0.37	0\\
0.371	0\\
0.372	0\\
0.373	0\\
0.374	0\\
0.375	0\\
0.376	0\\
0.377	0\\
0.378	0\\
0.379	0\\
0.38	0\\
0.381	0\\
0.382	0\\
0.383	0\\
0.384	0\\
0.385	0\\
0.386	0\\
0.387	0\\
0.388	0\\
0.389	0\\
0.39	0\\
0.391	0\\
0.392	0\\
0.393	0\\
0.394	0\\
0.395	0\\
0.396	0\\
0.397	0\\
0.398	0\\
0.399	0\\
0.4	0\\
0.401	0\\
0.402	0\\
0.403	0\\
0.404	0\\
0.405	0\\
0.406	0\\
0.407	0\\
0.408	0\\
0.409	0\\
0.41	0\\
0.411	0\\
0.412	0\\
0.413	0\\
0.414	0\\
0.415	0\\
0.416	0\\
0.417	0\\
0.418	0\\
0.419	0\\
0.42	0\\
0.421	0\\
0.422	0\\
0.423	0\\
0.424	0\\
0.425	0\\
0.426	0\\
0.427	0\\
0.428	0\\
0.429	0\\
0.43	0\\
0.431	0\\
0.432	0\\
0.433	0\\
0.434	0\\
0.435	0\\
0.436	0\\
0.437	0\\
0.438	0\\
0.439	0\\
0.44	0\\
0.441	0\\
0.442	0\\
0.443	0\\
0.444	0\\
0.445	0\\
0.446	0\\
0.447	0\\
0.448	0\\
0.449	0\\
0.45	0\\
0.451	0\\
0.452	0\\
0.453	0\\
0.454	0\\
0.455	0\\
0.456	0\\
0.457	0\\
0.458	0\\
0.459	0\\
0.46	0\\
0.461	0\\
0.462	0\\
0.463	0\\
0.464	0\\
0.465	0\\
0.466	0\\
0.467	0\\
0.468	0\\
0.469	0\\
0.47	0\\
0.471	0\\
0.472	0\\
0.473	0\\
0.474	0\\
0.475	0\\
0.476	0\\
0.477	0\\
0.478	0\\
0.479	0\\
0.48	0\\
0.481	0\\
0.482	0\\
0.483	0\\
0.484	0\\
0.485	0\\
0.486	0\\
0.487	0\\
0.488	0\\
0.489	0\\
0.49	0\\
0.491	0\\
0.492	0\\
0.493	0\\
0.494	0\\
0.495	0\\
0.496	0\\
0.497	0\\
0.498	0\\
0.499	0\\
0.5	0\\
0.501	0\\
0.502	0\\
0.503	0\\
0.504	0\\
0.505	0\\
0.506	0\\
0.507	0\\
0.508	0\\
0.509	0\\
0.51	0\\
0.511	0\\
0.512	0\\
0.513	0\\
0.514	0\\
0.515	0\\
0.516	0\\
0.517	0\\
0.518	0\\
0.519	0\\
0.52	0\\
0.521	0\\
0.522	0\\
0.523	0\\
0.524	0\\
0.525	0\\
0.526	0\\
0.527	0\\
0.528	0\\
0.529	0\\
0.53	0\\
0.531	0\\
0.532	0\\
0.533	0\\
0.534	0\\
0.535	0\\
0.536	0\\
0.537	0\\
0.538	0\\
0.539	0\\
0.54	0\\
0.541	0\\
0.542	0\\
0.543	0\\
0.544	0\\
0.545	0\\
0.546	0\\
0.547	0\\
0.548	0\\
0.549	0\\
0.55	0\\
0.551	0\\
0.552	0\\
0.553	0\\
0.554	0\\
0.555	0\\
0.556	0\\
0.557	0\\
0.558	0\\
0.559	0\\
0.56	0\\
0.561	0\\
0.562	0\\
0.563	0\\
0.564	0\\
0.565	0\\
0.566	0\\
0.567	0\\
0.568	0\\
0.569	0\\
0.57	0\\
0.571	0\\
0.572	0\\
0.573	0\\
0.574	0\\
0.575	0\\
0.576	0\\
0.577	0\\
0.578	0\\
0.579	0\\
0.58	0\\
0.581	0\\
0.582	0\\
0.583	0\\
0.584	0\\
0.585	0\\
0.586	0\\
0.587	0\\
0.588	0\\
0.589	0\\
0.59	0\\
0.591	0\\
0.592	0\\
0.593	0\\
0.594	0\\
0.595	0\\
0.596	0\\
0.597	0\\
0.598	0\\
0.599	0\\
0.6	0\\
0.601	0\\
0.602	0\\
0.603	0\\
0.604	0\\
0.605	0\\
0.606	0\\
0.607	0\\
0.608	0\\
0.609	0\\
0.61	0\\
0.611	0\\
0.612	0\\
0.613	0\\
0.614	0\\
0.615	0\\
0.616	0\\
0.617	0\\
0.618	0\\
0.619	0\\
0.62	0\\
0.621	0\\
0.622	0\\
0.623	0\\
0.624	0\\
0.625	0\\
0.626	0\\
0.627	0\\
0.628	0\\
0.629	0\\
0.63	0\\
0.631	0\\
0.632	0\\
0.633	0\\
0.634	0\\
0.635	0\\
0.636	0\\
0.637	0\\
0.638	0\\
0.639	0\\
0.64	0\\
0.641	0\\
0.642	0\\
0.643	0\\
0.644	0\\
0.645	0\\
0.646	0\\
0.647	0\\
0.648	0\\
0.649	0\\
0.65	0\\
0.651	0\\
0.652	0\\
0.653	0\\
0.654	0\\
0.655	0\\
0.656	0\\
0.657	0\\
0.658	0\\
0.659	0\\
0.66	0\\
0.661	0\\
0.662	0\\
0.663	0\\
0.664	0\\
0.665	0\\
0.666	0\\
0.667	0\\
0.668	0\\
0.669	0\\
0.67	0\\
0.671	0\\
0.672	0\\
0.673	0\\
0.674	0\\
0.675	0\\
0.676	0\\
0.677	0\\
0.678	0\\
0.679	0\\
0.68	0\\
0.681	0\\
0.682	0\\
0.683	0\\
0.684	0\\
0.685	0\\
0.686	0\\
0.687	0\\
0.688	0\\
0.689	0\\
0.69	0\\
0.691	0\\
0.692	0\\
0.693	0\\
0.694	0\\
0.695	0\\
0.696	0\\
0.697	0\\
0.698	0\\
0.699	0\\
0.7	0\\
0.701	0\\
0.702	0\\
0.703	0\\
0.704	0\\
0.705	0\\
0.706	0\\
0.707	0\\
0.708	0\\
0.709	0\\
0.71	0\\
0.711	0\\
0.712	0\\
0.713	0\\
0.714	0\\
0.715	0\\
0.716	0\\
0.717	0\\
0.718	0\\
0.719	0\\
0.72	0\\
0.721	0\\
0.722	0\\
0.723	0\\
0.724	0\\
0.725	0\\
0.726	0\\
0.727	0\\
0.728	0\\
0.729	0\\
0.73	0\\
0.731	0\\
0.732	0\\
0.733	0\\
0.734	0\\
0.735	0\\
0.736	0\\
0.737	0\\
0.738	0\\
0.739	0\\
0.74	0\\
0.741	0\\
0.742	0\\
0.743	0\\
0.744	0\\
0.745	0\\
0.746	0\\
0.747	0\\
0.748	0\\
0.749	0\\
0.75	0\\
0.751	0\\
0.752	0\\
0.753	0\\
0.754	0\\
0.755	0\\
0.756	0\\
0.757	0\\
0.758	0\\
0.759	0\\
0.76	0\\
0.761	0\\
0.762	0\\
0.763	0\\
0.764	0\\
0.765	0\\
0.766	0\\
0.767	0\\
0.768	0\\
0.769	0\\
0.77	0\\
0.771	0\\
0.772	0\\
0.773	0\\
0.774	0\\
0.775	0\\
0.776	0\\
0.777	0\\
0.778	0\\
0.779	0\\
0.78	0\\
0.781	0\\
0.782	0\\
0.783	0\\
0.784	0\\
0.785	0\\
0.786	0\\
0.787	0\\
0.788	0\\
0.789	0\\
0.79	0\\
0.791	0\\
0.792	0\\
0.793	0\\
0.794	0\\
0.795	0\\
0.796	0\\
0.797	0\\
0.798	0\\
0.799	0\\
0.8	0\\
0.801	0\\
0.802	0\\
0.803	0\\
0.804	0\\
0.805	0\\
0.806	0\\
0.807	0\\
0.808	0\\
0.809	0\\
0.81	0\\
0.811	0\\
0.812	0\\
0.813	0\\
0.814	0\\
0.815	0\\
0.816	0\\
0.817	0\\
0.818	0\\
0.819	0\\
0.82	0\\
0.821	0\\
0.822	0\\
0.823	0\\
0.824	0\\
0.825	0\\
0.826	0\\
0.827	0\\
0.828	0\\
0.829	0\\
0.83	0\\
0.831	0\\
0.832	0\\
0.833	0\\
0.834	0\\
0.835	0\\
0.836	0\\
0.837	0\\
0.838	0\\
0.839	0\\
0.84	0\\
0.841	0\\
0.842	0\\
0.843	0\\
0.844	0\\
0.845	0\\
0.846	0\\
0.847	0\\
0.848	0\\
0.849	0\\
0.85	0\\
0.851	0\\
0.852	0\\
0.853	0\\
0.854	0\\
0.855	0\\
0.856	0\\
0.857	0\\
0.858	0\\
0.859	0\\
0.86	0\\
0.861	0\\
0.862	0\\
0.863	0\\
0.864	0\\
0.865	0\\
0.866	0\\
0.867	0\\
0.868	0\\
0.869	0\\
0.87	0\\
0.871	0\\
0.872	0\\
0.873	0\\
0.874	0\\
0.875	0\\
0.876	0\\
0.877	0\\
0.878	0\\
0.879	0\\
0.88	0\\
0.881	0\\
0.882	0\\
0.883	0\\
0.884	0\\
0.885	0\\
0.886	0\\
0.887	0\\
0.888	0\\
0.889	0\\
0.89	0\\
0.891	0\\
0.892	0\\
0.893	0\\
0.894	0\\
0.895	0\\
0.896	0\\
0.897	0\\
0.898	0\\
0.899	0\\
0.9	0\\
0.901	0\\
0.902	0\\
0.903	0\\
0.904	0\\
0.905	0\\
0.906	0\\
0.907	0\\
0.908	0\\
0.909	0\\
0.91	0\\
0.911	0\\
0.912	0\\
0.913	0\\
0.914	0\\
0.915	0\\
0.916	0\\
0.917	0\\
0.918	0\\
0.919	0\\
0.92	0\\
0.921	0\\
0.922	0\\
0.923	0\\
0.924	0\\
0.925	0\\
0.926	0\\
0.927	0\\
0.928	0\\
0.929	0\\
0.93	0\\
0.931	0\\
0.932	0\\
0.933	0\\
0.934	0\\
0.935	0\\
0.936	1e-06\\
0.937	1e-06\\
0.938	1e-06\\
0.939	1e-06\\
0.94	1e-06\\
0.941	1e-06\\
0.942	1e-06\\
0.943	1e-06\\
0.944	1e-06\\
0.945	1e-06\\
0.946	1e-06\\
0.947	1e-06\\
0.948	1e-06\\
0.949	1e-06\\
0.95	1e-06\\
0.951	1e-06\\
0.952	1e-06\\
0.953	1e-06\\
0.954	1e-06\\
0.955	1e-06\\
0.956	1e-06\\
0.957	1e-06\\
0.958	1e-06\\
0.959	1e-06\\
0.96	1e-06\\
0.961	1e-06\\
0.962	1e-06\\
0.963	1e-06\\
0.964	1e-06\\
0.965	1e-06\\
0.966	1e-06\\
0.967	1e-06\\
0.968	1e-06\\
0.969	1e-06\\
0.97	2e-06\\
0.971	2e-06\\
0.972	2e-06\\
0.973	2e-06\\
0.974	2e-06\\
0.975	2e-06\\
0.976	2e-06\\
0.977	2e-06\\
0.978	2e-06\\
0.979	2e-06\\
0.98	2e-06\\
0.981	2e-06\\
0.982	2e-06\\
0.983	2e-06\\
0.984	2e-06\\
0.985	2e-06\\
0.986	2e-06\\
0.987	2e-06\\
0.988	2e-06\\
0.989	2e-06\\
0.99	2e-06\\
0.991	2e-06\\
0.992	2e-06\\
0.993	2e-06\\
0.994	2e-06\\
0.995	2e-06\\
0.996	2e-06\\
0.997	2e-06\\
0.998	2e-06\\
0.999	2e-06\\
1	2e-06\\
};
\addlegendentry{$t = \SI{11.2}{\nano\second}$}

\end{axis}
\end{tikzpicture}%
    \caption{Visualization of the displacement distribution in the bar for every position for longitudinal displacement input $r(t)=\si{1e-6}$ at $x=\SI{0}{\metre}$\\Source: \textsc{matlab}}
\end{figure}